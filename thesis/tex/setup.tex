\documentclass[normaltoc,espacoumemeio,pnumromarab,ruledheader]{abnt}     

\newcommand{\oautor}{Danilo Mutti}
\newcommand{\otitulo}{Coverage based debugging visualization}
\newcommand{\okw}{Code Cactus; Code Forest; Object Oriented Development; Debugging; Java3D; Java; Infovis}
%%%%%%%%%%%%%%%%%%%%%%%%%%%%%%%%%%%%%%%%%%%%%%%%%%%%%%%%%%%%%%%%%%%%%%%%%%%%%%%%%%%%%%%%%%%%%%%%%%%%

\usepackage[english]{babel}
\usepackage[latin1]{inputenc}
\usepackage{afterpage}
\usepackage{amsmath}
\usepackage{amstext}
\usepackage{amssymb}
\usepackage{theorem}
\usepackage{pbox}
\usepackage{fancyhdr}
\usepackage{epsf}
%\usepackage{graphicx}
\usepackage[tight,footnotesize]{subfigure}
\usepackage[linesnumbered,boxed,vlined,english]{algorithm2e}

\usepackage{rotating} % UTILIZADO PELO AMBIENTE SIDEWAYTABLE PARA TABELAS GIRADAS DE 90o
\usepackage{booktabs} % ALTERA O LAYOUT DE TABELAS
\usepackage{longtable} % CRIA TABELAS EM MAIS DE UMA P�GINA
\usepackage{float} % MELHORA O POSICIONAMENTO DE OBJETOS FLOAT COMO FIGURAS E TABELAS
\usepackage[format=hang,font=it,justification=centerlast,labelfont=bf,labelsep=endash]{caption} % FORMATA A LEGENDA DE TABELAS E FIGURAS

\usepackage{hhline}
\usepackage{epstopdf}
\usepackage{color, soul}
\usepackage{pgfgantt}
\usepackage{tabularx}
\usepackage{lscape}
\usepackage{longtable}
\usepackage{tabularx}
\usepackage{multicol,caption}
\usepackage{multirow}
\usepackage[normalem]{ulem}%http://www.ling.ohio-state.edu/~jonsafari/latex/
%documentacao em http://www.tug.org/applications/hyperref/manual.html
\usepackage[hidelinks]{hyperref}
\hypersetup{
  pdfauthor={\oautor},
  pdftitle={\otitulo},
  pdfsubject={Software Engineering},
  pdfkeywords={\okw},
  pdfproducer={Texlipse},
  pdfcreator={Texlipse}
}
\usepackage{pdfpages}
\usepackage{bookmark}
\usepackage{graphicx}
\usepackage{bchart}
\usepackage[output=pdf]{logo-each}

%%%%%%%%%%%%%%%%%%%%%%%%%%%%%%%%%%%%%%%%%%%%%%%%%%%%%%%%%%%%%%%%%%%%%%

% Additional ABNTex definitions.
%\usepackage[disable=copyright,disable=biblabel]{ach2017}

\newtheorem{theorem}{Teorema}
\newtheorem{acknowledgement}{Acknowledgement}
\newtheorem{axiom}{Axioma}
\newtheorem{case}{Caso}
\newtheorem{Propriedade}{Propriedade}
\newtheorem{claim}{Claim}
\newtheorem{conclusion}{Conclus\~ao}
\newtheorem{condition}{Condi\c{c}\~ao}
\newtheorem{conjecture}{Conjectura}
\newtheorem{corollary}{Corol\'ario}
\newtheorem{criterion}{Criterio}
\newtheorem{definition}{Defini\c{c}\~ao}
\newtheorem{example}{Exemplo}
\newtheorem{exercise}{Exercise}
\newtheorem{lemma}{Lema}
\newtheorem{notation}{Notat\c{c}\~ao}
\newtheorem{problem}{Problema}
\newtheorem{proposition}{Proposi\c{c}\~ao}
\newtheorem{remark}{Observa\c{c}\~ao}
\newtheorem{solution}{Solu\c{c}\~ao}
\newtheorem{summary}{Sum\'ario}
\newenvironment{proof}[1][Demonstra\c{c}\~ao]{\textbf{#1.} }{\ \rule{0.5em}{0.5em}\vspace{0.5cm}}
\newcommand{\usp}{Universidade de S\~{a}o Paulo}
\newcommand{\MakeIndex}{\textsl{MakeIndex}}
\newcommand{\itemnorma}[1]{\textit{#1}}
\newcommand{\ingles}[1]{\textsl{#1}}
\newcommand{\bibTeX}{bib\kern-.13ex\TeX}%
\newcommand{\abnt}{{\smaller ABNT}}%
\newcommand{\report}{{\smaller REPORT}}%
\newcommand{\xypic}{X\hspace*{-.2ex}\raisebox{-.5ex}{Y}-pic}%
 %\newcommand{\ac}{\symbol{123}}  % abre chaves
%\newcommand{\fc}{\symbol{125}}  % fecha chaves
\newcommand{\bs}{\symbol{92}}   % barra invertida (backslash)
\renewcommand{\ABNTtravessao}{}
\setlength{\ABNTanapindent}{0cm}
\renewcommand{\ABNTaposindicativoanap}
                    {\protect\\[4mm]\protect\centering}

\hyphenation{le-vels for-est}


% ***************************************
% *******Capa personalizada *************
% ***************************************



%---------------------------------------------------------------------

\providecommand{\ABNTvarautordata}{}
\let\oldautor\autor\relax
\renewcommand{\autor}[2][] {%
\renewcommand{\ABNTvarautordata}{#1}
\oldautor{#2}}

\renewcommand{\capa} {%
\begin{titlepage}
\makeheader
\begin{center}
\vfill
{\Large\ABNTautordata}\\[2cm]
{\LARGE\textbf{\ABNTtitulodata}}\\[2cm]
\vfill
{\large \ABNTlocaldata \\ \ABNTdatadata}
\end{center}
\end{titlepage}}

% ***************************************
% ****Folha de rosto personalizada ******
% ***************************************


\renewcommand{\folhaderosto}{
\begin{titlepage}
    \vfill
    \begin{center}
        {\large \ABNTautordata}\\[5cm]
        {\LARGE \textbf{\ABNTtitulodata}}\\[5cm]
        \hspace{.45\textwidth}
        \begin{minipage}{.5\textwidth}
        \begin{espacosimples}
            \begin{small}
                \ABNTcomentariodata
                \\ \\
                %�rea de Concentra��o: \ABNTareadata
%As duas linhas abaixo devem ser comentadas se n�o houver necessidade de op��o
                %\\
                %Op��o: \ABNTopcaodata
%--------------------------------------------------------------------------------------------
                \\
                Orientador: \ABNTorientadordata
%As duas linhas abaixo devem ser comentadas se n�o houver necessidade de co-orientador
%                \\
%                Co-orientador: \ABNTcoorientadordata
%--------------------------------------------------------------------------------------------
            \end{small}
        \end{espacosimples}
        \end{minipage}
        \vfill
        {\large \ABNTlocaldata \\ \ABNTdatadata}
    \end{center}
\end{titlepage}
}

%************************************
%**** Defini��es das ref�rencias*****
%************************************

%\usepackage[alf,abnt-and-type=&]{abntcite}
%\setlength{\headheight}{15pt}
\usepackage[alf,abnt-etal-cite=3,abnt-etal-list=0]{abntcite}
\usepackage{abnt-UFPR}
\newcommand*{\refname}{References}

% **************** Meus Comandos *********************
\newcommand{\superscript}[1]{\ensuremath{^{\textrm{#1}}}}
\newcommand{\hlc}[2][yellow]{ {\sethlcolor{#1} \hl{#2}} }
\newenvironment{Figure}{\par\medskip\noindent\minipage{\linewidth}}{\endminipage\par\medskip}